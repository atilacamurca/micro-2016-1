\section{Introdução}\label{introduuxe7uxe3o}

O Jogo Frets On Fire tem o objetivo de simular que o jogador seja um
astro do rock. A intenção é que ele possa tocar músicas simulando
instrumentos como guitarra, baixo e bateria.

Esse jogo usa o Teclado do computador como entrada das notas musicais. A
bateria utiliza 5 comandos do teclado.

Sendo assim é possível criar uma plataforma de entrada de dados usando
um microcontrolador e piezos elétricos para captar as batidas do usuário
e enviá-las para o jogo, ou seja, transformar sinais elétricos em
comandos do teclado.

\section{Material Necessário}\label{material-necessuxe1rio}

\begin{itemize}
\itemsep1pt\parskip0pt\parsep0pt
\item
  5 Piezos Elétricos
\item
  1 Arduino
\item
  5 Resistores 
\item
  Fios
\item
  1 Breadboard
\item
  1 par de Baquetas
\item
  7 folhas de EVA (2 pretas, 1 vermelha, 1 amarela, 1 azul, 1 verde)
\item
  Pistola e bastões de cola quente
\end{itemize}

\section{Objetivos}\label{objetivos}

O objetivo desse projeto é captar as batidas do usuário através dos
piezos, enviar um sinal como se fosse o teclado relacionando o piezo a
uma tecla.

O jogo pode ser configurado para receber qualquer tecla. Com isso é
posível até rodar o jogo em modo \emph{multiplayer}, sem que haja
conflito de teclas. O jogador que irá simular a guitarra ou o baixo pode
utilizar as teclas:

\begin{itemize}
\itemsep1pt\parskip0pt\parsep0pt
\item
  \texttt{1} (Nota Verde),
\item
  \texttt{2} (Nota Vermelha),
\item
  \texttt{3} (Nota Amarela),
\item
  \texttt{4} (Nota Azul),
\item
  \texttt{5} (Nota Laranja),
\item
  \texttt{/} (Palheta 1),
\item
  \texttt{*} (Palheta 2),
\item
  \texttt{-} (\emph{Star Power})
\item
  e \texttt{8} (Alavanca);
\end{itemize}

enquanto o jogador da bateria pode utilizar as teclas:

\begin{itemize}
\itemsep1pt\parskip0pt\parsep0pt
\item
  \texttt{a} (Nota Vermelha),
\item
  \texttt{e} (Nota Amarela),
\item
  \texttt{h} (Nota Azul),
\item
  \texttt{j} (Nota Verde),
\item
  \texttt{\textless{}espaço\textgreater{}} (Pedal),
\end{itemize}

por exemplo.

\section{Cronograma}\label{cronograma}

\begin{longtable}[c]{@{}lllll@{}}
\toprule\addlinespace
\begin{minipage}[b]{0.39\columnwidth}\raggedright
Tarefas
\end{minipage} & \begin{minipage}[b]{0.12\columnwidth}\raggedright
Semana 1
\end{minipage} & \begin{minipage}[b]{0.12\columnwidth}\raggedright
Semana 2
\end{minipage} & \begin{minipage}[b]{0.12\columnwidth}\raggedright
Semana 3
\end{minipage} & \begin{minipage}[b]{0.12\columnwidth}\raggedright
Semana 4
\end{minipage}
\\\addlinespace
\midrule\endhead
\begin{minipage}[t]{0.39\columnwidth}\raggedright
Aquisição do Material
\end{minipage} & \begin{minipage}[t]{0.12\columnwidth}\raggedright
X
\end{minipage} & \begin{minipage}[t]{0.12\columnwidth}\raggedright
\end{minipage} & \begin{minipage}[t]{0.12\columnwidth}\raggedright
\end{minipage} & \begin{minipage}[t]{0.12\columnwidth}\raggedright
\end{minipage}
\\\addlinespace
\begin{minipage}[t]{0.39\columnwidth}\raggedright
Protótipo Inicial
\end{minipage} & \begin{minipage}[t]{0.12\columnwidth}\raggedright
X
\end{minipage} & \begin{minipage}[t]{0.12\columnwidth}\raggedright
\end{minipage} & \begin{minipage}[t]{0.12\columnwidth}\raggedright
\end{minipage} & \begin{minipage}[t]{0.12\columnwidth}\raggedright
\end{minipage}
\\\addlinespace
\begin{minipage}[t]{0.39\columnwidth}\raggedright
Montagem do circuito
\end{minipage} & \begin{minipage}[t]{0.12\columnwidth}\raggedright
\end{minipage} & \begin{minipage}[t]{0.12\columnwidth}\raggedright
X
\end{minipage} & \begin{minipage}[t]{0.12\columnwidth}\raggedright
\end{minipage} & \begin{minipage}[t]{0.12\columnwidth}\raggedright
\end{minipage}
\\\addlinespace
\begin{minipage}[t]{0.39\columnwidth}\raggedright
Programação
\end{minipage} & \begin{minipage}[t]{0.12\columnwidth}\raggedright
\end{minipage} & \begin{minipage}[t]{0.12\columnwidth}\raggedright
X
\end{minipage} & \begin{minipage}[t]{0.12\columnwidth}\raggedright
X
\end{minipage} & \begin{minipage}[t]{0.12\columnwidth}\raggedright
\end{minipage}
\\\addlinespace
\begin{minipage}[t]{0.39\columnwidth}\raggedright
Testes
\end{minipage} & \begin{minipage}[t]{0.12\columnwidth}\raggedright
\end{minipage} & \begin{minipage}[t]{0.12\columnwidth}\raggedright
\end{minipage} & \begin{minipage}[t]{0.12\columnwidth}\raggedright
X
\end{minipage} & \begin{minipage}[t]{0.12\columnwidth}\raggedright
\end{minipage}
\\\addlinespace
\begin{minipage}[t]{0.39\columnwidth}\raggedright
Relatório
\end{minipage} & \begin{minipage}[t]{0.12\columnwidth}\raggedright
X
\end{minipage} & \begin{minipage}[t]{0.12\columnwidth}\raggedright
\end{minipage} & \begin{minipage}[t]{0.12\columnwidth}\raggedright
\end{minipage} & \begin{minipage}[t]{0.12\columnwidth}\raggedright
X
\end{minipage}
\\\addlinespace
\bottomrule
\addlinespace
\caption{Cronograma}
\end{longtable}

\section{Referências}\label{referuxeancias}

\begin{itemize}
\itemsep1pt\parskip0pt\parsep0pt
\item
  Frets On Fire
\item
  FoFiX
\item
  Arduino
\item
  Arduino IDE
\end{itemize}
